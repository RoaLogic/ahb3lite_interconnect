\chapter{Configuration}\label{configuration}


\section{Introduction}\label{introduction-1}


The Roa Logic AHB-Lite Multi-layer Interconnect is a highly configurable
interconnect fabric for AMBA AHB-Lite based systems. The core parameters
and configuration options are described in this section.

\section{Core Parameters}\label{core-parameters}

\begin{longtable}[]{@{}lccp{5cm}@{}}
\toprule
Parameter & Type & Default & Description\tabularnewline
\midrule
\endhead
\texttt{HADDR\_SIZE}           & Integer & 32     & Address bus size\tabularnewline
\texttt{HDATA\_SIZE}           & Integer & 32     & Data bus size\tabularnewline
\texttt{MASTERS}               & Integer & 3      & Number of master ports\tabularnewline
\texttt{SLAVES}                & Integer & 8      & Number of slave ports\tabularnewline
\texttt{SLAVE\_MASK[MASTERS]}  & Array  & All '1's & Mask slaves accessible by each master\tabularnewline
\texttt{ERROR\_ON\_SLAVE\_MASK[MASTERS]} & Array & inv(\texttt{SLAVE\_MASK}) & Enable error reporting for masked slaves\tabularnewline
\texttt{ERROR\_ON\_NO\_SLAVE[MASTERS]}  & Array  & All '0's & Disable error reporting when non-mapped address space accessed (to match previous IP releases)\tabularnewline

\bottomrule
\caption{Core Parameters}
\end{longtable}

\subsection{HADDR\_SIZE}\label{haddr_size}

The \texttt{HADDR\_SIZE} parameter specifies the width of the address bus for all
master and slave ports.

\subsection{HDATA\_SIZE}\label{hdata_size}

The \texttt{HDATA\_SIZE} parameter specifies the width of the data bus for all
master and slave ports.

\subsection{MASTERS}\label{masters}

The \texttt{MASTERS} parameter specifies the number of master ports on the
interconnect fabric.

\subsection{SLAVES}\label{slaves}

The \texttt{SLAVES} parameter specifies the number of slave ports on the
interconnect fabric.

\subsection{SLAVE\_MASK[ ]}\label{slave_mask}

The \texttt{SLAVE\_MASK[\,]} parameter determines if a master may access a slave. Defining which
master may access individual slaves (rather than allowing all masters to access all slaves) may
significantly reduce the logic area of the interconnect and improve overall performance.

There is one \texttt{SLAVE\_MASK} parameter per master, each \texttt{SLAVES} bits wide.
i.e. \texttt{SLAVE\_MASK[\,]} is an array of dimensions \texttt{MASTERS} x \texttt{SLAVES}.

Setting a \texttt{SLAVE\_MASK[\,]} bit to '0' indicates that master cannot access the slave.
Conversely, setting a \texttt{SLAVE\_MASK[\,]} bit to '1' indicates that master may access the slave.

\subsection{ERROR\_ON\_SLAVE\_MASK[ ]}\label{error_on_slave_mask}

The \texttt{ERROR\_ON\_SLAVE\_MASK[\,]} parameter enables generating an AHB error response when the
master attempts to access a masked slave port.

There is one \texttt{ERROR\_ON\_SLAVE\_MASK} parameter per master, each \texttt{SLAVES} bits wide.
i.e. \texttt{ERROR\_ON\_SLAVE\_MASK[\,]} is an array of dimensions \texttt{MASTERS} x \texttt{SLAVES}.

Setting an \texttt{ERROR\_ON\_SLAVE\_MASK[\,]} bit to '0' indicates that an AHB error response will not be generated if the master is masked from accessing the corresponding slave.
Conversely, setting a \texttt{ERROR\_ON\_SLAVE\_MASK[\,]} bit to '1' indicates that an AHB error response will
be generated if the master is masked from accessing the corresponding slave.

The default value of \texttt{ERROR\_ON\_SLAVE\_MASK[\,]} is the bitwise inverse of \texttt{SLAVE\_MASK[\,]} - i.e. inv(\texttt{SLAVE\_MASK[\,]}). If \texttt{SLAVE\_MASK[\,]} is assigned a value, then \texttt{ERROR\_ON\_SLAVE\_MASK[\,]} is by default inv(\texttt{SLAVE\_MASK[\,]}).

\subsection{ERROR\_ON\_NO\_SLAVE[ ]}\label{error_on_no_slave}

The AHB-Lite Multi-layer Interconnect uses \texttt{slv\_addr\_base} and \texttt{slv\_addr\_mask} to decode the target slave port. It can therefore determine if a master attempts to access a non-mapped slave port. This means the switch can generate an AHB error response should a master attempt to access an address which is not mapped to any slave port.

The \texttt{ERROR\_ON\_NO\_SLAVE[\,]} is a is \texttt{MASTERS} bits wide parameter used to enable error response generation for this scenario. Setting a bit of the parameter to '1' enables error generation for the corresponding master. 

The default value for \texttt{ERROR\_ON\_NO\_SLAVE[\,]} is all bits set to '0', disabling this feature to match the behaviour of previous releases of the IP.

\section{Core Macros}\label{core-macros}

\begin{longtable}[]{@{}ll@{}}
\toprule
Macro & Description\tabularnewline
\midrule
\endhead
\texttt{RECURSIVE\_FUNCTIONS\_SUPPORTED} & Enable use of recursive functions \tabularnewline
\bottomrule
\caption{Core Macros}
\end{longtable}

\subsection{RECURSIVE\_FUNCTIONS\_SUPPORTED}\label{recursive_functions_supported}

Recursive functions and modules are used within the verilog source code. However EDA tools vary in their support for recursion, with some supporting recursive functions, whereas others support recursive modules or both. For example, Intel Quartus v19.1 and earlier supports recursive modules, but not recursive functions. 

To accomodate these toolchain differences, recursive modules are the default method used for implementation. However recursive functions may instead be enabled by setting the synthesis macro \texttt{RECURSIVE\_FUNCTIONS\_SUPPORTED}
